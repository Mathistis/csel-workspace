\documentclass{ReportTemplate}
\usepackage{titlesec}
\title{CSEL}
\author{Macherel Rémy}
\date{\today}
\subtitle{Rapports des TP}
\subsubtitle{Git du projet : \href{https://github.com/Mathistis/csel-workspace}{https://github.com/Mathistis/csel-workspace}}
\location{Fribourg,}
\contact{remy.macherel@master.hes-so.ch}
\version{1.0}
\titlespacing*{\chapter}{0pt}{-60pt}{20pt}
\begin{document}

\maketitlepage

\newpage

\maketableofcontent

\medskip

\titleformat{\chapter}[display]
    {\Huge\bfseries}
    {}
    {0pt}
    {\thechapter.\ }
    

\chapter{TP Systèmes de fichiers}

\section{Résumé du travail pratique}
Ce travail consiste à réaliser une application qui contrôle la fréquence de
clignotement d'une LED sur la carte NanoPi. L'application doit permettre grâce
aux systèmes de fichiers et en passant par les boutons de la board de régler la
fréquence de la LED. Un programme de base (silly\_led\_control.c) est fourni
mais celui-ci consomme l'entièreté d'un coeur du processeur. Il nous est donc
demandé de développer un programme qui utilise les systèmes de fichiers pour
contrôler la LED ainsi que les boutons et qui fonctionnerait de manière plus optimisée.

\section{Infos utiles à retenir}
Pour obtenir le numéro de GPIO correspondant à une pin, il faut lire la
configuration à l'aide de la commande :
\begin{minted}{shell}
    mount -t debugfs none /sys/kernel/debug
    cat /sys/kernel/debug/gpio
\end{minted}
Afin d'utiliser des event produits par les gpio associés aux boutons, il est
nécessaire d'utiliser \textit{EPOLLERR} car en utilisant \textit{EPOLLIN} cela
ne fonctionne pas. Nous n'avons pas réellement trouvé la raison de ceci et
suspectons un bug dans le kernel. %TODO: Expliquer mieux
Pour les boutons, au moment du open sur le fd, il faut faire un pread pour
quittancer l'interruption alors que pour le timer le read suffit.
\section{Feedback global}
Nous avons dû retrouver dans un ancien TP les numéros de pin des boutons de la
carte car en utilisant les commandes de la section précédente, si ceux-ci ne
sont pas activés cela ne les affiche pas.

\chapter{TP 5}
\section{Résumé du laboratoire}


\section{Réponses aux questions}

\section{Synthèse des connaissances acquises}
\subsection{Non acquis}

\subsection{Acquis, mais à exercer}

\subsection{Parfaitement acquis}

\section{Feedback exercices}


\chapter{TP 6}
\section{Résumé du laboratoire}

\section{Synthèse des connaissances acquises}
\subsection{Non acquis}

\subsection{Acquis, mais à exercer}

\subsection{Parfaitement acquis}

\section{Feedback}

\end{document}


