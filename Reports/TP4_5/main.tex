\documentclass{ReportTemplate}
\usepackage{titlesec}
\title{CSEL}
\author{Macherel Rémy}
\date{\today}
\subtitle{Rapports des TP}
\subsubtitle{Git du projet : \href{https://github.com/Mathistis/csel-workspace}{https://github.com/Mathistis/csel-workspace}}
\location{Fribourg,}
\contact{remy.macherel@master.hes-so.ch}
\version{1.0}
\titlespacing*{\chapter}{0pt}{-60pt}{20pt}
\begin{document}

\maketitlepage

\newpage

\maketableofcontent

\medskip

\titleformat{\chapter}[display]
    {\Huge\bfseries}
    {}
    {0pt}
    {\thechapter.\ }
    

\chapter{TP Systèmes de fichiers}

\section{Résumé du travail pratique}
Ce travail consiste à réaliser une application qui contrôle la fréquence de
clignotement d'une LED sur la carte NanoPi. L'application doit permettre grâce
aux systèmes de fichiers et en passant par les boutons de la board de régler la
fréquence de la LED. Un programme de base (silly\_led\_control.c) est fourni
mais celui-ci consomme l'entièreté d'un coeur du processeur. Il nous est donc
demandé de développer un programme qui utilise les systèmes de fichiers pour
contrôler la LED ainsi que les boutons et qui fonctionnerait de manière plus optimisée.

\section{Infos utiles à retenir}
Pour obtenir le numéro de GPIO correspondant à une pin, il faut lire la
configuration à l'aide de la commande :
\begin{minted}{shell}
    mount -t debugfs none /sys/kernel/debug
    cat /sys/kernel/debug/gpio
\end{minted}
Afin d'utiliser des event produits par les gpio associés aux boutons, il est
nécessaire d'utiliser \textit{EPOLLERR} car en utilisant \textit{EPOLLIN} cela
ne fonctionne pas. Nous n'avons pas réellement trouvé la raison de ceci et
suspectons un bug dans le kernel. %TODO: Expliquer mieux
Pour les boutons, au moment du open sur le fd, il faut faire un pread pour
quittancer l'interruption alors que pour le timer le read suffit.
\section{Feedback global}
Nous avons dû retrouver dans un ancien TP les numéros de pin des boutons de la
carte car en utilisant les commandes de la section précédente, si ceux-ci ne
sont pas activés cela ne les affiche pas.

\chapter{TP 5}
\section{Résumé du laboratoire}
Le but de ce laboratoire est de mettre en pratique la théorie vue dans le cours
en ce qui concerne le multiprocessing ainsi que les ordonnanceurs. Il se compose
de deux exercices, un sur les processus et signaux de communication et l'autre
sur l'utilisation des CGroups. Les réponses aux questions ainsi que les
difficultés rencontrées ou information utiles à retenir seront détaillés dans
plus loin dans ce document.

\section{Réponses aux questions}
\subsection{Quel effet a la commande echo \$\$ > ... sur les cgroups ?}
L'utilisation des caractères \textit{\$\$} permet d'obtenir le PID du processus
actuel et donc dans notre cas du shell duquel sera lancé le processus. Cette
commande permet donc de définir les limites de ressources que pourra utiliser
notre processus. Il est utile de préciser que tout processus enfant du shell
(correspondant au PID \$\$) auront les mêmes limites.
\subsection{Quel est le comportement du sous-système memory lorsque le quota de mémoire est épuisé ? Pourrait-on le modifier ? Si oui, comment ?}

\subsection{Est-il possible de surveiller/vérifier l’état actuel de la mémoire ? Si oui, comment ?}

\section{Synthèse des connaissances acquises}
\subsection{Non acquis}

\subsection{Acquis, mais à exercer}

\subsection{Parfaitement acquis}

\section{Feedback exercices}
\subsection{Exercice 2}
Il a été observé que si l'on lance le programme et que l'on essaie de limiter le
nombre de bytes possibles alors qu'il a déjà atteint le quota, l'écriture est
bloquée et la limite ne peut être fixée. Cependant si l'on écrit la limite avant
que le programme ne l'ait atteinte, elle sera bien fixée et le programme
s'arrêtera lorsqu'il atteindra cette limite.\newline
Commande pour limiter la quantité max de mémoire :
\begin{minted}[linenos=false]{shell}
echo 20M > /sys/fs/cgroup/set/program/memory.limit_in_bytes
\end{minted}

Nous avons également pu observer que si l'on crée deux cgroup de limitation de
mémoire, un de 20M et l'autre de 30M. Si l'on ajoute notre PID dans celui de 20M
puis dans celui de 30M, au moment de l'ajout dans celui de 30M il est
automatiquement retiré des tasks du cgroup qui limitait à 20M.

\chapter{TP 6}
\section{Résumé du laboratoire}

\section{Synthèse des connaissances acquises}
\subsection{Non acquis}

\subsection{Acquis, mais à exercer}

\subsection{Parfaitement acquis}

\section{Feedback}

\end{document}


